%%Énoncé
Quelles sont les trois méthodes principales du type abstrait \texttt{Map} ? Qu’entend-on par clé,
valeur et entry dans ce contexte ? Est-il possible d’insérer une valeur null dans un \texttt{Map} ? Expliquez. \\
Citez plusieurs exemples d’application d’un \texttt{Map} et précisez dans chaque cas à quel type d’information 
correspond chaque \textit{entry}.

%%Réponse
TODO after ma conférence 
Ebauche de réponse:

Les trois méthodes principales d'un type abstrait \texttt{Map} \textbf{M} sont les suivantes:
\begin{enumerate}
\item \textit{get(k)}: Si M contient une entrée e = (k,v), avec une clé égale à k, alors la valeur de retour sera v, pour e 
ou alors renverra \textbf{null}.
\item \textit{put(k,v)}: Si M n'a pas d'entrée avec une clé égale à k, alors il faut ajouter une entrée (k,v) à M pour 
que l'on retourne \textbf{null}; sinon, il faut remplacer v par la valeur existante de l'éntrée avec la clé égale à k et retourner 
la valeur antérieure.
\item \textit{remove(k)}:
\end{enumerate}
